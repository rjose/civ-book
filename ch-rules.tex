\chapter{Rules}
Coding can be fun and rewarding, but learning to code can be discouraging in
part because there are so many decisions to make. Should we start with the user
interface and then move down or should we start with the underlying components
and move up? What are the inputs to our application? How does the input get in?
What form should it have? What are the outputs? How does it get to the user?
What should we do when things go wrong? What errors do we need to handle? Will
we have to store data? Where should we put it? How should we organize it? How
should we model our data? How shouldwe name it? How do the elements of our
application interact with each other? Are there well-defined states? How do we
transition between them? Should we use built-in data structures or create
custom ones? Should we write tests first? How should we comment code? Should we
write big functions or lots of little ones? Should we do object-oriented
programming or functional programming (or both)? Should we indent with spaces
or tabs? The list of decisions stretches on forever.

But even after we've made our decisions, how do we know we've made good ones?
Are we heading down a difficult or dead-end path? Will we have to reach the end
before we have to head back and start over? How many dead ends will we have to
take before we've coded something that works well?

Fortunately, the coders who came before us have figured out what are generally
good ideas over decades of trial and error. They've figured out what choices
tend to yield good results and which ones lead to problems. These ideas and
choices have become rules for how to code.

Coding rules establish guidelines that help us get started. They allow us to
focus on the problems we want to solve instead of puzzling over a myriad of
choices. They define concepts and metaphors for thinking about coding problems
and the patterns for solving them. They define a framework for thinking about
problems and solutions. When we can map our problems into a framework, we get
solutions from the framework for free.

% TODO: I wonder if we can bring the category theory ideas in here to formalize
% this notion of mapping problems into frameworks and viewing different
% perspectives almost like different categories?

Rules can keep us out of trouble by bounding safe areas to work in. When we
stay within these boundaries, we avoid certain kinds of problems. When we stick
to proven patterns, our coding experience can be more fruitful and
enjoyable---we solve problems at hand speedily and with confidence.

\section{Evolution of Rules}

Rules start from ideas that worked well, practices that prevented a class of
mistakes or that made coding more productive. They are established within dev
teams and if generally effective, spread to the industry at large and if
universally effective to the software field itself.

\subsection{Team Rules}

Rules start in teams as generally good ideas for systematically addressing
problems. Some rules address common but avoidable errors. For instance, in C++,
a vector data structure stores objects in a contiguous space of memory and
provides a way to access them using an index. But in C++, it's common to
refer to objects by a pointer to their memory location, making it very natural
to use pointers to objects in vectors. This can lead to a problem that's
disastrous and very hard to identify and reproduce: when the vector runs out of
memory, it will allocate another block of memory and copy all of its elements
and discard its previous memory store, turning all previous pointers into
garbage. A team that runs into this will probably create a rule like ``Never
refer to vector elements by pointers---always use indexes''.

Some rules are conventions that don't necessarily improve the code but make it
easier for people to work together. Conventions for how to name things, for
instance, can make it easier for everyone to understand code without extra
documentation. A team may decide that all functions private to a module are
lower case and those used outside of a module are capitalized, or that all
class variables start with an underscore, or that all constants are
capitalized. These conventions create consistency within code, making it easier
for everyone to read, and also to write, because this set of decisions has
already been made.

Process conventions form another class of rules that focus on how code is
developed. For instance, when code is checked into a version control system,
there is a main branch, the trunk, which contains official versions of the
code, and there are variations of the trunk rooted at different spots called
branches. One process convention is for coders to work on their own branches
and then merge their work back in when it's time to release a new version.
While this approach has the benefit of allowing people to work quickly in
parallel, it can result in a ``big merge'' problem when several people have
changed the same parts of code unintentionally change behavior that others rely
on. A big merge can take a lot of time and effort to resolve, and worse, it may
introduce subtle bugs or inadvertently undo bug fixes. A team that ships a bad
merge will probably come up with process rules like ``All changes go onto the
trunk'' or ``Only one person can work on a module at one time''.

Over time, teams create many rules and conventions. Teams with track records
for quality code and efficient execution become the models for other teams in a
company. Their rules spread and become adopted throughout the org.

\subsection{Industry Rules}

Some rules are specific to particular domains or industries (e.g.,
semiconductor equipment, medical devices, web development, AI). These industry
rules are adopted because they address problems that everyone in that industry
eventually runs into. In numerical code, people have come up with rules that
result in higher performance code that preserves precision as much as possible,
rules like ``Consolidate additions and subtractions before multiplications'' and
``Sum a long sequence of numbers by ordering them with alternating signs of
similar magnitude''.

Some rules are conventions that everyone in that industry follows. In web
development, every response from a server has a status code whose values
indicate something that every web dev understands: codes between 200 and 299
mean that the request was ok, the specific numbers give additional information;
codes between 400 and 499 mean that the client made a mistake in constructing
the request; codes between 500 and 599 mean that the server made a mistake,
typically not handling an exception properly. Anyone who does web work is
intimately familiar with these rules and can write code that uses them to
integrate smoothly with any web server or client application.

\subsection{Field Rules}

Some rules transcend all industries and are generally accepted across the
software field. A well-known example is Dijkstra's classic 1968 note to the
ACM: ``Go To Statement Considered Harmful'' where he argued that goto makes
programs hard to follow and understand and overusing goto can ``make a mess of
one's program''. The goto statement is a necessary part of assembly language
coding, but in higher level languages it can lead to code that is difficult to
reason about: the number of ways of reaching a section of code is unbounded and
unpredictable. This rule has been so embraced by coders that even the phrase
``\textless something\textgreater\ considered harmful'' flags something as an
anathema within the software field.

Another field rule is ``Don't use global variables'' (variables accessible
throughout the code and that can be changed by anyone at any time during code
execution). Like ``goto'', overusing global variables can lead to code that is
difficult to reason about because it isn't clear how and when the values of
global variables were changed. But, more seriously, is when two pieces of code
are in conflict over what the value of a global variable should be. If one
piece of code sets a global to one value and another changes it before the
first is done using it, the program will surely be in a bad state where all
subsequent results are suspect. Tracking down problems like these can be a
nightmare.

Some field rules take the form of advice that saves time. For instance, Knuth's
``...premature optimization is the root of all evil...'' is a well-known
example. The problem with optimizing too soon is that it involves adding
complexity to the code, which may or may not help depending on where the
performance issues actually are. Coders are notoriously bad at guessing at what
will be expensive because we think in terms of what would be time consuming for
us to do rather than the computer. It turns out that the computer is super fast
at a lot of things. When we add complexity to a program, we make it harder to
understand and harder to change later on. Worse, it may force the code to take
on a certain design or even architecture depending on the nature of the
optimization.  A follow on rule to Knuth's adage is ``profile before
optimizing''. By profile, we mean to analyze the runtime behavior of the
software to see where it spends the most time and where it makes the most
function calls. These are the areas focus on. Often, the place to optimize is
an inocuous part of the code that no one would ever suspect of being a problem.

Some field rules are process conventions that are widely accepted almost as
standards. For instance, the ``Agile'' methodology has steadily grown from its
conception in 2001 by a number of software's thought leaders to the de facto
standard process for most software teams. Today, no one will bat an eye if
someone asks when the daily stand-up meeting is or be puzzled if someone asks
how many story points they think something will take or how long a sprint
should be. Most coders today are very comfortable working in an Agile
environment because they already know its rules.

\section{Rules Establish Culture}

Rules establish the concepts, vocabulary, and metaphors that form the culture of
teams, industries, and the software field. Concepts like data structures and
algorithms are some of the first things coders learn. Terms like ``array'' and
``queue'' have very specific meanings and well-defined operations. The ``tree''
metaphor is ubiquitious and all of its associated terms are familiar to any
coder: root, branches, leaves.

Vocabulary is at the heart of culture. It is so ingrained in the minds of
coders that even close synonyms of standard terms seem like foreign concepts.
The term ``heap'' refers to a data structure whose operations maintain a
particular ordering of its elements. But the very similar words ``pile'' and
``mound'' don't really mean anything to a coder and seeing them in code would
certainly require explanation, if not justification.

Becoming a coder means learning the rules of software, and learning the rules
is a social activity. We learn the rules of the field from teachers and
professors, from classmates and mentors. We learn them by reading what people
post online. We learn them when we go to meetups and conferences, when we watch
presentations and talk to attendees and speakers. We learn them in our
internships and from our coworkers when we work professionally.

Even when we write code alone, we learn from the designers of the languages we
use. Their choice of names and syntax reflect the rules they felt were
important. Many popular languages don't have a ``goto'' statement or global
variables, and so when you code in these languages, you internalize those
particular rules implicitly. Through programming languages, designers reinforce
industry concepts and vocabulary, providing ways to frame coding problems and
solve them.

But it is on teams where rules serve their most important social function. On
teams, rules help coders work together without misunderstanding. If code must
always be peer reviewed before it can be checked in, no one is surprised if
they can't check code in at 2 am. If code reviews are rigorous, no one is upset
when they receive critical feedback. If the team has rules on how variables are
named or functions are commented, then code that violates these rules will look
incorrect even if nothing is technically wrong. Rules define the customs and
establish the culture of a team.

The job interview is essentially an assessment of a candidate's compatibility
with and understanding of the rules of the field, industry, and team. When a
recruiter calls, they are checking that you know the vocabulary of the field
and industry. The phone screen makes sure you understand concepts. The lunch
interview helps figure out how compatible you are with the team's rules. And
while coding and design interviews are intended to gauge your problem-solving
skills, they are also assessing how well you understand the rules of the field
and industry. It's entirely possible for you to solve a problem, but
fail the interview because you didn't use established terms or didn't use them
correctly (for instance, if you said that you'd put the elements in a
``pile''). Conversely, if you have a solid grasp of the rules, you might answer
a question in a few words without any whiteboarding at all. By using the right
terms in the right context, you can immediately establish a shared
understanding with the interviewing (e.g., ``You could use
two heaps with one as a priority queue'').

\section{Rules Define a Perspective}

The rules of a team, industry, or field are really perspectives on how work
should be done. Rules implicitly define a value system on what makes good code.
Some rules like ``Don't optimize too soon'' are nearly universal, but others
depend on the problems that a team or industry faces and may actually conflict
with each other. Code that runs on an interstellar probe must be extremely
robust. Code for pacemakers must handle all possible failure modes safely. Code
for a web startup needs to be developed and deployed (and fixed) quickly. Code
used to prototype a product concept needs to be more flexible than robust.

If you've worked in several industries, you've probably picked up several
perspectives (and languages) and may find it easy to adopt new ones because
you understand that rules are different because of different problems. But if
you've only worked within one domain and have only ever used on language, then
you've really only coded from one perspective and are certain to have implicit
biases. If you stay in one industry for your entire career, this might not
matter---you can certainly have a rich, promotion-filled career doing this. But
your progression as a coder will be stunted. After all, how can you master
coding when you only know one way to do it?
